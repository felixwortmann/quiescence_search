%!TEX root = ../dokumentation.tex

\chapter{Einleitung}
Diese Hausarbeit beschäftigt sich mit dem Thema \textit{Quiescence Seach} oder auch \textit{Ruhesuche}. Sie wurde im Kontext der Vorlesung "`Wissensbasierte Systeme"', gehalten an der DHBW Mannheim von Prof. Dr. Karl Stroetmann, erstellt. Diese Arbeit sowie die Implementierungen -- alle relevanten Teile werden nichtsdestotrotz in dieser Arbeit dargestellt -- lassen sich auf \textit{GitHub} unter folgendem Link abrufen: \url{https://github.com/felixwortmann/quiescence_search}.

\section{Aufgabenstellung und -kontext}
Ziel dieser Arbeit ist es, das Verfahren der Quiescence Search zu erläutern und eine Implementierung im Rahmen einer Schach-KI (Künstliche Intelligenz) zu demonstrieren. Diese KI wurde im Rahmen einer Studienarbeit, ebenfalls betreut von Prof. Dr. Karl Stroetmann, entwickelt und basiert auf dem \textit{Minimax-Algorithmus} sowie auf dem \textit{Alpha-Beta-Pruning} und der \textit{Memoization}. Bevor auf den Quiescence Search Algorithmus eingegangen wird, werden jene Implementierungen gezeigt und erläutert sowie die Theorie, auf der sie aufbauen, erklärt. Schließlich wird auf die vorhandene KI die Quiescence Search aufgebaut und implementiert, sodass als Resultat dieser Arbeit ein funktionierender Algorithmus entsteht, welcher mithilfe von Quiescence Search eine Verbesserung zu dem bestehenden darstellt.