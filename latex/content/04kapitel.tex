%!TEX root = ../dokumentation.tex

\chapter{Fazit und Ausblick}
Zusammenfassend lässt sich sagen, dass Programme wie die entwickelte Schach-KI stets viele Ansatzpunkte für Verbesserungen haben. Die Verbesserung in Form der Quiescence Search wurde in dieser Arbeit analysiert. Auch, wenn die Erweiterung zur Folge hat, dass die Berechnung der einzelnen Züge länger dauert, handelt es sich bei dem Horizonteffekt um ein Problem mit starken Auswirkungen, weshalb die Implementierung einer Quiescence Search von sehr großem Vorteil für die generelle Qualität der errechneten Züge ist.

Als mögliche Weiterentwicklung der Implementierung bietet sich an, eine größere Menge an taktischen Zügen zu definieren. Schließlich kann der Horizonteffekt potentiell bei jedem Zug auftreten; auch hier ist es eine Abwägung von Performanz-Einbußungen und Zug-Verbesserungen.

Generell war die Implementierung der Ruhesuche erfolgreich und hat die generelle Qualität der Schach-KI verbessert. Im Vergleich zum gesamten Programm des Minimax-Algorithmus handelt es sich bei der Ruhesuchenerweiterung um einen eher kleinen Teil, welcher jedoch große Auswirkungen hat. Dies spricht zweifellos für die Verwendung einer Quiescence Search bei derartigen Programmen.